%%%%%%%%%%%%%%%%%%%%%%%%%%%%%%%%%%%%%%%%%%%%%%%%%%
%%                                              %%
%%        Comandos usados no documento          %%
%%                                              %%
%%%%%%%%%%%%%%%%%%%%%%%%%%%%%%%%%%%%%%%%%%%%%%%%%%
% Capa e Folha de Rosto
\renewcommand{\imprimircapa}{
%%%%%%%%%%%%%%%%%%%%%%%%%%%%%%%%%%%%%%%%%%%%%%%%%%
%%                                              %%
%%              Início da Capa                  %%
%%                                              %%
%%%%%%%%%%%%%%%%%%%%%%%%%%%%%%%%%%%%%%%%%%%%%%%%%%

\thispagestyle{empty}

\begin{figure}
    \centering
    \includegraphics[]{lib/Logoufrn.jpg}
\end{figure}

\begin{center}
    \large{\imprimirinstituicao}
    
    \vspace{\stretch{2}}
    
    \large{\MakeUppercase{\imprimirtitulo: \imprimirsubtitulo}}
    
    \vspace{\stretch{2}}
    
    \large{\MakeUppercase{\textbf{\imprimirautor}}}
    
    \vspace{\stretch{3}}
    
    \large{\imprimirlocal} \\
    \large{\imprimirdata}
\end{center}

\newpage
%%%%%%%%%%%%%%%%%%%%%%%%%%%%%%%%%%%%%%%%%%%%%%%%%%
%%                                              %%
%%                  Fim da Capa                 %%
%%                                              %%
%%%%%%%%%%%%%%%%%%%%%%%%%%%%%%%%%%%%%%%%%%%%%%%%%%}
\renewcommand{\imprimirfolhaderosto}{
%%%%%%%%%%%%%%%%%%%%%%%%%%%%%%%%%%%%%%%%%%%%%%%%%%
%%                                              %%
%%          Início da Folha de Rosto            %%
%%                                              %%
%%%%%%%%%%%%%%%%%%%%%%%%%%%%%%%%%%%%%%%%%%%%%%%%%%

\thispagestyle{empty}

\begin{center}
    % Imprime o nome do autor depois coloca um espaçamento vertical de 3 vezes o espaçamento existente entre os textos
    \large{\MakeUppercase{\textbf{\imprimirautor}}}
    \vspace*{\stretch{3}}
    
    % Imprime o título do trabalho depois coloca um espaçamento vertical padrão dentre os textos da página
    \large{\MakeUppercase{\imprimirtitulo: \imprimirsubtitulo}}
    \vspace{\stretch{1}}
    
    % Inicia a mini página alinhada horizontalmente com o comprimento da linha sendo a metade da mesma imprimindo o preâmbulo, o orientador do trabalho e o co-orientador (se houver). Após a mini página o espaçamento vertical
    \hfill
    \begin{minipage}{.5\linewidth}
        \small\imprimirpreambulo
        \\ [0.6cm]
        \small \imprimirorientadorRotulo \imprimirorientador.\\
        \small \imprimircoorientadorRotulo
        \small \imprimircoorientador.
    \end{minipage}
    \vspace{\stretch{1}}
    
    % Imprime local e data do trabalho
    \large{\imprimirlocal}\\
    \large{\imprimirdata}
\end{center}

\newpage
%%%%%%%%%%%%%%%%%%%%%%%%%%%%%%%%%%%%%%%%%%%%%%%%%%
%%                                              %%
%%            Fim da Folha de Rosto             %%
%%                                              %%
%%%%%%%%%%%%%%%%%%%%%%%%%%%%%%%%%%%%%%%%%%%%%%%%%%
}

% subtítulo
\providecommand{\imprimirsubtitulo}{}
\newcommand{\subtitulo}[1]{\renewcommand{\imprimirsubtitulo}{#1}}


% criar fonte em imagens, tabelas e quadros
\newcommand{\source}[1]{\legend{\textbf{Fonte:} {#1}}}    

% texto em arial
%\renewcommand{\sfdefault}{phv}
%\renewcommand{\rmdefault}{phv}

% texto em Times
%\renewcommand{\sfdefault}{ptm}
%\renewcommand{\rmdefault}{ptm}

%%%%%%%%%%%%%%%%%%%%%%%%%%%%%%%%%%%%%%%%%%%%%%%%%%
%%                                              %%
%%          Informações do documento            %%
%%                                              %%
%%%%%%%%%%%%%%%%%%%%%%%%%%%%%%%%%%%%%%%%%%%%%%%%%%

\titulo{\textbf{Aprendizado semissupervisionado aplicado à classificação}}

\subtitulo{um estudo de caso com o \textit{FlexCon\hyp{C}} para análise da variação do limiar}

\tipotrabalho{Monografia (bacharel em Sistema de Informação)}

\autor{Arthur Costa Gorg\^{o}nio}

\orientador{ MSc. Amarildo Jeiele Ferreira de Lucena}     % nível acadêmico + nome

\coorientador{ MSc. Karliane Medeiros Ovidio Vale}

%%%%%%%%%%%%%%%%%%%%%%%%%%%%
%% Pacotes com Hierarquia %%
%%%%%%%%%%%%%%%%%%%%%%%%%%%%

% Rótulo do orientador
\renewcommand{\imprimirorientadorRotulo}{Orientador(a): }
\renewcommand{\imprimircoorientadorRotulo}{Co-orientador(a): }

\local{Caicó - RN}

\data{\the\year}

\instituicao{
    UNIVERSIDADE FEDERAL DO RIO GRANDE DO NORTE \\
    CENTRO DE ENSINO SUPERIOR DO SERIDÓ \\
    DEPARTAMENTO DE COMPUTAÇÃO E TECNOLOGIA \\
    BACHARELADO EM SISTEMAS DE INFORMAÇÃO
}

%% Se for TCC II acrescente mais um I
\preambulo{\textbf{Trabalho de Conclusão de Curso I}, apresentado ao Curso de Bacharelado em Sistemas de Informação da Universidade Federal do Rio Grande do Norte, como parte dos requisitos para obtenção do título de Bacharel em Sistemas de Informação.}


%%%%%%%%%%%%%%%%%%%%%%%%%%%%%%%%%%
%%                              %%
%%          Numeração           %%
%%                              %%
%%%%%%%%%%%%%%%%%%%%%%%%%%%%%%%%%%
\counterwithout{equation}{chapter}

%%%%%%%%%%%%%%%%%%%%%%%%%%%%%%%%%%
%%                              %%
%%            Quadros           %%
%%                              %%
%%%%%%%%%%%%%%%%%%%%%%%%%%%%%%%%%%
\newcommand{\quadroname}{Quadro}

\newfloat[chapter]{quadro}{loq}{\quadroname}
\newlistof{listofquadros}{loq}{\listofquadrosname}
\newlistentry{quadro}{loq}{0}

% configurações para atender às regras da ABNT
%\setfloatadjustment{quadro}{\centering}
\counterwithout{quadro}{chapter}
\renewcommand{\cftquadroname}{\quadroname\space} 
\renewcommand*{\cftquadroaftersnum}{\hfill--\hfill}

%\setfloatlocations{quadro}{}


%%%%%%%%%%%%%%%%%%%%%%%%%%%%%%%%%%%%%%%%%%%%%%%%%%
%%                                              %%
%%   Configurações de aparência do PDF final    %%
%%                                              %%
%%%%%%%%%%%%%%%%%%%%%%%%%%%%%%%%%%%%%%%%%%%%%%%%%%

%%%%%%%%%%%%%%%%%%%%%%%%%%%%%%%%%%
%%                              %%
%%         Indentação           %%
%%                              %%
%%%%%%%%%%%%%%%%%%%%%%%%%%%%%%%%%%

% O tamanho do parágrafo é dado por:
\setlength{\parindent}{1.5cm}

% Controle do espaçamento entre um parágrafo e outro:
\setlength{\parskip}{0.2cm}  % tente também \onelineskip


%%%%%%%%%%%%%%%%%%%%%%%%%%%%%%%%%%
%%                              %%
%%      Informações do PDF      %%
%%                              %%
%%%%%%%%%%%%%%%%%%%%%%%%%%%%%%%%%%
\makeatletter
\hypersetup{
	pdftitle={\@title}, 
	pdfauthor={\@author},
    pdfsubject={\imprimirpreambulo},
    pdfcreator={LaTeX with abnTeX2},
	pdfkeywords={abnt}{latex}{abntex}{abntex2}{trabalho acadêmico}, 
%   false: boxed links; true: colored links
	colorlinks=true,
% 	color of internal links
    linkcolor=blue,
%   color of links to bibliography
    citecolor=blue,
%   color of file links
    filecolor=magenta,
	urlcolor=blue,
	bookmarksdepth=4
}
\makeatother

% Algoritmos
% \floatname{algorithm}{Algoritmo}%Algoritmo
% \renewcommand{\listalgorithmname}{LISTA DE ALGORITMOS}

\renewcommand{\algorithmicindent}{3.0em}
\renewcommand{\algorithmicrequire}{\textbf{entrada}}
\renewcommand{\algorithmicensure}{\textbf{garanta}}
\renewcommand{\algorithmicreturn}{\textbf{retorne}}
\renewcommand{\algorithmicor}{\textbf{ou}}
\renewcommand{\algorithmicand}{\textbf{e}}
\renewcommand{\algorithmicend}{\textbf{fim}}
\renewcommand{\algorithmicif}{\textbf{se}}
\renewcommand{\algorithmicelse}{\textbf{sen\~ao}}
\renewcommand{\algorithmicthen}{\textbf{ent\~ao}}
\renewcommand{\algorithmicfor}{\textbf{para}}
\renewcommand{\algorithmicforall}{\textbf{para cada}}
\renewcommand{\algorithmicrepeat}{\textbf{repita}}
\renewcommand{\algorithmicuntil}{\textbf{até que}}
\renewcommand{\algorithmicwhile}{\textbf{enquanto}}
\renewcommand{\algorithmicdo}{\textbf{fa\c{c}a}}

