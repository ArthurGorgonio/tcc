\section{Discussão}
    \label{sec:discussao}

    % No trabalho de \citeonline{vale2018selftraining} o algoritmo \ac{flexcon} teve o parâmetro $cr$ pré\hyp{fixado} em 5\%, neste trabalho foi analisada a influência deste parâmetro na acurácia da classificação de dados semissupervisionados. No experimento proposto, foram utilizadas 31 bases de dados, aplicando variações no parâmetro $cr$ em até três pontos percentuais num intervalo discreto.

    Na Seção~\ref{sec:effectiveness-analysis} foram apresentados resultados que o método \ac{flexcon} está sujeito a diversas variações de resultados a partir dos valores selecionados para cada um dos parâmetros. Entretanto, quando se pré\hyp{fixa} os demais parâmetros observa\hyp{se} que nem sempre o valor utilizado por \citeonline{vale2018selftraining}, $cr$ pré\hyp{fixado} em 5\%, obtém os melhores resultados para cada uma das técnicas estudadas. Neste trabalho, foi analisada a influência do parâmetro $cr$ na acurácia da classificação de dados semissupervisionados.