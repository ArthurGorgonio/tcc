% Neste Tópico é onde o aluno irá discorrer exatamente o problema no qual a sua pesquisa pretende resolver, com base nos objetivos apontados no Capítulo 1.

\section{Questões de Pesquisa}
    \label{sec:questoes-de-pesquisa}

    % Comparativo de métodos
    % variação da taxa do limiar
    % 3%, 5% e 7%
    % Métricas de avaliação (comparação com 5%)
    % Analisar os algoritmos e explicar?
    %  que foram propostos em~\citeonline{vale2018selftraining} são: FlexConf-C1 e FlexConf-C2. 

    Como explanado na Subseção~\ref{subsec:objetivo-geral}, este trabalho conduziu uma análise no comportamento do método \ac{flexcon}, realizando manipulações exclusivamente na taxa de variação do limiar, mantendo a proposta original discutida em~\cite{vale2018selftraining}, taxa de variação fixa em 5\%. Na presente pesquisa, foi avaliada a influência do parâmetro que varia o limiar na classificação de dados. A faixa de variação do parâmetro considerada foi entre 2\% e 8\%, comparando os resultados obtidos de cada um destes valores com o valor base 5\%.

    A validação da proposta deste trabalho foi por meio de uma comparação das médias das acurácias e variâncias obtidas pelo respectivo classificador. Para realizar uma análise do ponto de vista estatístico foi aplicado o teste de \textit{Friedman}, pois esse ignora a suposição de normalidade dos dados. \citeonline{friedman1937test} definiu um teste estatístico que realiza um ranqueamento dos dados ao invés de utilizar os dados brutos, para evitar a suposição da normalidade. Esse tipo de análise pode ser empregada no lugar da análise de variância quando há mais de um critério de classificação. A hipótese nula desse teste afirma que o efeito da variável-resposta é nulo entre as populações testadas.

    Ao aplicar o teste, os dados foram convertidos em uma matriz M x N onde, M corresponde aos blocos que representam os indivíduos da população e N corresponde as diversas condições/tratamentos~\cite{pereira2015friedman}.

    % Esse teste recebe uma matriz M x N onde, M corresponde aos blocos que representam os indivíduos da população e N corresponde as diversas condições/tratamentos~\cite{pereira2015friedman}.

    % realizada a análise de variância (do inglês, \ac{anova}) que é um teste de hipótese que valida se duas ou mais populações são iguais. 

    % Para avaliação do ponto de vista estatístico, o teste de \textit{Friedman and Nemenyi Post-Hoc} foi selecionado para essa comparação, por realizar análise de variância não paramétrica bidirecional~\cite{norheim1986friedman}.

    % Como métricas de avaliação será desenvolvida uma comparação dos algoritmos a partir média das acurácias obtidas nas bases de dados com um dos classificadores alterando apenas a variação que é incorporada ao limiar. Essa análise será realizada com o teste estatístico de \textit{Friedman and Quade}, por ser uma análise de variância não paramétrica bidirecional.