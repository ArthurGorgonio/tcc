\section{Delimitação do Estudo}
    \label{sec:delimitacao-estudo}

    % Algoritmos
    % Classificadores
    % Parâmetros
    % , uma vez que os autores que desenvolveram esses métodos utilizaram a taxa de 5\%

    Neste trabalho serão analisados os efeitos da taxa de variação do limiar sobre o método \ac{flexcon} do \ac{ssl}, uma vez que ele possui como restrição a variação do limiar pré\hyp{fixada} em 5\%. Para isso, apenas a taxa de variação do limiar será modificada ao longo da execução dos experimentos, mantendo-se inalterados os demais parâmetros do algoritmo. Assim como apresentado em~\citeonline{vale2018selftraining}, também serão utilizados os mesmos percentuais de exemplos inicialmente rotulados(5\%, 10\%, 15\%, 20\% e 25\%) e classificadores (Na\"ive Bayes).

   Na implementação das técnicas propostas \textit{FlexCon\hyp{C1}} e \textit{FlexCon\hyp{C2}}, foram utilizados quatro diferentes classificadores, que possibilitam a exploração dos diversos tipos de dados existentes nas bases de dados, a saber: baseado em método Bayesiano (Na\"ive Bayes), baseado em Árvore de Decisão (\textit{rpartXse}), baseado em Instâncias (\ac{knn}) e baseado em Regras de Associação (\ac{ripper}). Essas técnicas foram submetidas a um conjunto de 31 bases de dados, realizando uma comparação entre a média das acurácias obtidas pelas técnicas quando aplicada a proposta.