\section{Objetivos}
    \label{sec:objetivos}

    Neste tópico são apresentados os objetivos desta pesquisa, divididos em geral e específicos.

    \subsection{Objetivo Geral}
        \label{subsec:objetivo-geral}
        % Este trabalho tem por objetivo aplicar as variações do algoritmo \textit{Self\hyp{Training}} sendo elas \textit{FlexConf\hyp{C1}} e \textit{FlexConf\hyp{C2}} manipulando a variação da taxa de confiança e avalia\hyp{las}.

        % Este trabalho tem por objetivo analisar dois dos métodos propostos em~\citeauthor{vale2018selftraining} e manipular a taxa de variação do limiar de tais métodos.

        %Este trabalho tem por objetivo analisar algoritmos do \ac{ssl}  com a finalidade de avaliá\hyp{los}, no que diz respeito à sua eficácia e acurácia na classificação dos dados, manipulando dinamicamente a taxa de variação do limiar de tais métodos e aplicando\hyp{os} em diversas bases de dados.

        % Este trabalho tem por objetivo manipular dinamicamente a taxa de variação do limiar, analisando dois algoritmos do \ac{ssl} com a finalidade de avaliá\hyp{los}, no que diz respeito à sua eficácia e acurácia na classificação dos dados e aplicando\hyp{os} em diversas bases de dados.

        % Este trabalho tem por objetivo alterar a taxa de variação do limiar e analisar os seus efeitos em no algoritmo do \ac{ssl} proposto em \citeonline{vale2018selftraining}, o \ac{flexcon}, em suas duas variações \textit{FlexCon\hyp{C1}} e o \textit{FlexCon\hyp{C2}}, para a tarefa de classificação de dados semissupervisionados.

        Este trabalho tem por objetivo analisar a influência da taxa de variação do limiar na acurácia da classificação de dados semissupervisionados no método proposto em \citeauthor{vale2018selftraining}, o \ac{flexcon}, em suas duas variações \textit{FlexCon\hyp{C1}} e o \textit{FlexCon\hyp{C2}}.

    \subsection{Objetivos Específicos}
        \label{subsec:objetivos-especificos}
        \begin{enumerate}[label=\alph*)]
            \item Definir os valores mínimos e máximos da variação do limiar; % 2% até 8%
            \item Propor métricas que permitam avaliar os experimentos a serem realizados; % comparação pelo teste de fridman and quade
            \item Selecionar um conjunto de bases de dados para a aplicação dos experimentos;
            \item Implementar os algoritmos, modificando-os para inclusão das variações descritas na proposta;
            
            \item Validar os resultados obtidos através da comparação empírica dos algoritmos modificados.
        \end{enumerate}